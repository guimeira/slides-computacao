\documentclass{beamer}
\usetheme[faculty=med]{fibeamer}
\usepackage[utf8]{inputenc}
\usepackage[
  main=english
]{babel}        %% typeset as follows:
%% These macros specify information about the presentation
\title{Algoritmos} %% that will be typeset on the
\subtitle{Condicionais} %% title page.
\author{Guilherme Meira}
%% These additional packages are used within the document:
\usepackage{ragged2e}  % `\justifying` text
\usepackage{booktabs}  % Tables
\usepackage{tabularx}
\usepackage{tikz}      % Diagrams
\usetikzlibrary{calc, shapes, backgrounds, positioning}
\usepackage{minted}
\usepackage{amsmath, amssymb}
\usepackage{url}       % `\url`s
\usepackage{listings}  % Code listings
\usepackage{xcolor}
\definecolor{highlightcolor}{RGB}{255, 140, 119}
\setminted{highlightcolor=highlightcolor}
\frenchspacing
\begin{document}
  \frame{\maketitle}
  \AtBeginSection[]{% Print an outline at the beginning of sections
  	\begin{frame}<beamer>
  		\frametitle{Agenda}
  		\tableofcontents[currentsection]
  	\end{frame}}
\section{Operadores relacionais}
\begin{frame}{Operadores relacionais}
	Como nossos programas podem tomar decisões?
	\begin{itemize}
		\item Se a média final é menor que 5, reprovado
		\item Se a temperatura está acima de 25 graus, ligar o ar condicionado
		\item Se a senha informada está correta, pode entrar no sistema
	\end{itemize}
\end{frame}
\begin{frame}{Operadores relacionais}
	Podemos comparar valores utilizando \alert{operadores relacionais}:
	\begin{itemize}
		\item \textbf{Igual:} \texttt{==}
		\item \textbf{Diferente:} \texttt{!=}
		\item \textbf{Maior que:} \texttt{>}
		\item \textbf{Menor que:} \texttt{<}
		\item \textbf{Maior ou igual:} \texttt{>=}
		\item \textbf{Menor ou igual:} \texttt{<=}
	\end{itemize}
\end{frame}
\begin{frame}{Operadores relacionais}
	Os operadores relacionais retornam:
	\begin{itemize}
		\item \textbf{Zero} se a comparação for falsa
		\item \textbf{Um} se a comparação for verdadeira
	\end{itemize}
	Em C:
	\begin{itemize}
		\item \textbf{Zero} é considerado \alert{falso}
		\item \textbf{Qualquer outro valor} é considerado \alert{verdadeiro}
	\end{itemize}
\end{frame}
\begin{frame}{Operadores relacionais}
	\inputminted{c}{resources/relational.c}
\end{frame}
\begin{frame}{Operadores relacionais}
	\begin{block}{Saída}
		\inputminted{text}{resources/relational.txt}
	\end{block}
\end{frame}
\section{Condicionais}
\begin{frame}{Condicionais}
	Para tomarmos uma decisão, usamos o \alert{\texttt{if}}:
	\inputminted{c}{resources/if.c}
\end{frame}
\begin{frame}{Condicionais}
	Também podemos executar código caso a condição dentro do \texttt{if} seja falsa. Para isso, usamos o \alert{\texttt{else}}.
	\inputminted[fontsize=\footnotesize]{c}{resources/else.c}
\end{frame}
\begin{frame}{Condicionais}
	Podemos construir sequências de \texttt{if} e \texttt{else}.
	\inputminted[fontsize=\footnotesize]{c}{resources/ifsequence.c}
\end{frame}
\begin{frame}{Condicionais}
	Podemos usar \texttt{if} dentro de outro \texttt{if}.
	\inputminted[fontsize=\footnotesize]{c}{resources/ifnested.c}
\end{frame}
\begin{frame}{Condicionais}
	Atenção à \textbf{indentação}!
	\begin{itemize}
		\item Sempre que abrir chaves, passe a escrever mais à frente usando espaços ou TABs
		\item A maioria dos editores de código fazem isso automaticamente para você
		\item Melhora a legibilidade do seu código
		\item \textbf{Vale ponto na prova}
	\end{itemize}
\end{frame}
\begin{frame}{Condicionais}
	\begin{exampleblock}{Certo}
		\inputminted[fontsize=\footnotesize]{c}{resources/identation.c}
	\end{exampleblock}
	\begin{exampleblock}{Errado}
		\inputminted[fontsize=\footnotesize]{c}{resources/identationwrong.c}
	\end{exampleblock}
\end{frame}
\begin{frame}{Condicionais}
	\framesubtitle{Exercício 1}
	Escreva um programa que leia a nota de um aluno pelo teclado e informe se ele passou direto.
\end{frame}
\begin{frame}{Condicionais}
	\framesubtitle{Exercício 1}
	\inputminted[fontsize=\footnotesize]{c}{resources/ex1.c}
\end{frame}
\begin{frame}{Condicionais}
	\framesubtitle{Exercício 2}
	Escreva um programa que leia a nota de um aluno pelo teclado e:
	\begin{itemize}
		\item Se o aluno passou direto, imprima uma mensagem
		\item Se o aluno ficou de prova final, leia a nota da prova final pelo teclado e calcule a média final, informando se ele passou ou reprovou
	\end{itemize}
\end{frame}
\begin{frame}{Condicionais}
	\framesubtitle{Exercício 2}
	\only<1>{\inputminted[fontsize=\footnotesize,firstline=1,lastline=13]{c}{resources/ex2.c}}
	\only<2>{\inputminted[fontsize=\footnotesize,firstline=14]{c}{resources/ex2.c}}
\end{frame}
\begin{frame}{Condicionais}
	\framesubtitle{Exercício 3}
	Escreva um programa que leia uma letra do teclado e imprima uma palavra que comece com aquela letra.
\end{frame}
\begin{frame}{Condicionais}
	\framesubtitle{Exercício 3}
	\inputminted[fontsize=\footnotesize]{c}{resources/ex3.c}
\end{frame}
\begin{frame}{Condicionais}
	Para esses tipos de situação, existe o comando \alert{\texttt{switch}}.
	\only<1>{\inputminted[fontsize=\footnotesize]{c}{resources/switch.c}}
	\only<2>{\inputminted[fontsize=\footnotesize,highlightlines={7,11,15}]{c}{resources/switch.c}}
\end{frame}
\begin{frame}{Condicionais}
	O comando \alert{\texttt{switch}} pula para o \alert{\texttt{case}} que seja igual ao valor passado para ele e executa todo o código dali para frente, até encontrar um \alert{\texttt{break}}.
	\inputminted[fontsize=\footnotesize]{c}{resources/break.c}
\end{frame}
\begin{frame}{Condicionais}
	Podemos usar um valor especial chamado \alert{\texttt{default}}, caso nenhum \texttt{case} seja igual ao valor passado para o \texttt{switch}.
	\inputminted[fontsize=\footnotesize]{c}{resources/default.c}
\end{frame}
\begin{frame}{Condicionais}
	\framesubtitle{Exercício 4}
	Escreva um programa que leia um número pelo teclado e imprima o mês do ano correspondente ao número.
\end{frame}
\begin{frame}{Condicionais}
	\framesubtitle{Exercício 4}
	\only<1>{\inputminted[fontsize=\footnotesize,lastline=15]{c}{resources/ex4.c}}
	\only<2>{\inputminted[fontsize=\footnotesize,firstline=16]{c}{resources/ex4.c}}
\end{frame}
\begin{frame}{Condicionais}
	\framesubtitle{Exercício 5}
	Escreva um programa que leia uma temperatura e verifique se a água está líquida na temperatura informada.
\end{frame}
\begin{frame}{Condicionais}
	\framesubtitle{Exercício 5}
	\inputminted[fontsize=\scriptsize]{c}{resources/ex5.c}
\end{frame}
\begin{frame}{Condicionais}
	Para essas situações, podemos construir condições mais complexas utilizando os \alert{operadores lógicos}.
\end{frame}
\section{Operadores lógicos}
\begin{frame}{Operadores lógicos}
	Em C, temos três operadores lógicos:
	\begin{itemize}
		\item \textbf{E:} \texttt{\&\&}
		\item \textbf{Ou:} \texttt{||}
		\item \textbf{Não:} \texttt{!}
	\end{itemize}
\end{frame}
\begin{frame}{Operadores lógicos}
	O operador \textbf{e} retorna \textbf{verdadeiro} se \alert{ambas as condições forem verdadeiras}.
	\begin{table}[!b]
	{\carlitoTLF
		\begin{tabular}{ccc}
			\textbf{c1} & \textbf{c2} &  \textbf{c1 \&\& c2} \\
			\toprule
			V & V & V \\
			V & F & F \\
			F & V & F \\
			F & F & F \\
			\bottomrule
		\end{tabular}}
	\end{table}
\end{frame}
\begin{frame}{Operadores lógicos}
	O operador \textbf{ou} retorna \textbf{verdadeiro} se \alert{qualquer uma das condições for verdadeira}.
	\begin{table}[!b]
	{\carlitoTLF
		\begin{tabular}{ccc}
			\textbf{c1} & \textbf{c2} &  \textbf{c1 || c2} \\
			\toprule
			V & V & V \\
			V & F & V \\
			F & V & V \\
			F & F & F \\
			\bottomrule
		\end{tabular}}
	\end{table}
\end{frame}
\begin{frame}{Operadores lógicos}
	O operador \textbf{não} transforma \textbf{falso} em \textbf{verdadeiro} e \textbf{verdadeiro} em \textbf{falso}.
	\begin{table}[!b]
	{\carlitoTLF
		\begin{tabular}{cc}
			\textbf{c1} & \textbf{!c1} \\
			\toprule
			V & F \\
			F & V \\
			\bottomrule
		\end{tabular}}
	\end{table}
\end{frame}
\begin{frame}{Operadores lógicos}
	Solução do Exercício 5 com o uso de operadores lógicos:
	\inputminted[fontsize=\footnotesize]{c}{resources/logicops.c}
\end{frame}
\begin{frame}{Operadores lógicos}
	\framesubtitle{Exercício 6}
	Escreva um programa que leia a idade de uma pessoa pelo teclado e diga se para ela o voto é opcional.
	
	O voto é opcional para:
	\begin{itemize}
		\item Adolescentes de 16 e 17 anos
		\item Idosos acima de 70 anos
	\end{itemize}
\end{frame}
\begin{frame}{Operadores lógicos}
	\framesubtitle{Exercício 6}
	\inputminted[fontsize=\scriptsize]{c}{resources/ex6.c}
\end{frame}
\begin{frame}{Operadores lógicos}
	Podemos usar parênteses para construir condições mais complexas:
	\inputminted[firstline=8,lastline=8]{c}{resources/ex6.c}
	Neste exemplo, não precisaríamos de parênteses, pois \textbf{o operator \alert{\texttt{\&\&}} tem precedência sobre o operador \alert{\texttt{||}}}.
	
	Na prática, é bom utilizar os parênteses para tornar o código mais legível.
\end{frame}
\end{document}
